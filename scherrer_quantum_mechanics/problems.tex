\documentclass[9pt]{report}
\usepackage{graphicx}
\usepackage[utf8]{inputenc}
\usepackage{gensymb}
%\usepackage[T1]{fontenc}
\usepackage{textcomp}
%\usepackage[dutch]{babel}
\usepackage{amsmath, amssymb}

\begin{document}
\title{Scherrer's Quantum Mechanics\protect\\ Problems}
\author{Anthony Steel}
\date{\today}
\maketitle
\chapter{The Origin Of Quantum Mechanics}
  \begin{enumerate}
    \item \textbf{Assume that a human body emits blackbody radiation at the
      standard body temperature.}
      \begin{enumerate}
        \item \textbf{Estimate how much energy is radiated by the body in one hour.}

          The power emmitted by a blackbody is given by:
          \[
            P = \sigma A T^4
          \]
          where $A$ is the surface area of the body, $T$ is the temperature
          and $\sigma$ is the Stefan-Boltzmann constant. Therefore
        the energy radiated by a body in a given time interval $\Delta t$:
          \[
            E = \sigma A T^4 \Delta t
          \]
          The surface area of the human body is approximately $2\text{m}^2$,
          the average body temperature is $36.1^{\circ}\text{C} = 309.25 \text{K}$,
          and there are $3600$s in an hour. Therefore:
          \[
            \begin{align}
            E_\text{hour}
            &= \Big(5.67 * 10^{-8} \frac{\text{J}}{\text{s m}^2\text{K}^4}\Big)( 2\text{m}^2 )(309.25\text{K})^4(3600\text{s})\\
            &= 3733 \text{kJ}
            \end{align}
          \]

        \item \textbf{At what wavelength does this radiation reach a maximum}

          The formula for the maximum wavelength is:
          \[
            \lambda_\text{peak} = \frac{w}{T}
          \]
          where $w = 2.90 * 10^{-3} \text{m K}$ and $T = 309.25\text{K}$ as
          before. Therefore the maximum wavelength is:
          \[
            \lambda_\text{peak} = \frac{2.9 * 10^{-3} \text{m K}}{309.25\text{K}} = 9.37 * 10^{-6} \text{m}
          \]
      \end{enumerate}
        \item \textbf{A distant red star is observed to have a blackbody spectrum with a
          maximum at a wavelength of $3500$\AA [$1$\AA = $10^{-10}$ m]. What is the
          temperature of the star?}

          Inverting the formula from the pervious question:
          \[
            T = \frac{w}{\lambda_\text{peak}}
          \]
          giving:
          \[
            T = \frac{2.9 * 10^{-3} \text{m K}}{ 3500 * 10^{-10} \text{m}} = 51428 \text{K}
          \]

        \item \textbf{The universe is filled with blackbody radiation at a temperature
            of $2.7$K left over from the Big Bang. [This radiation was disvoered in
          1965 by Bell Laboratory scientists who thought at one point that they were seeing
          interference from pigeon droppings on their microwave reciever.}
          \begin{enumerate}
          \item
            \textbf{What is the total energy density of this radiation?}

            The  energy density of the radiation is given by:
              \[
                \rho = a T^4
              \]
              where $a = 7.56 * 10^{-16} \frac{\text{J}}{\text{m}^3 \text{K}^4}$.
              Therefore:
              \[
                \rho = 7.56 * 10^{-16} \frac{\text{J}}{\text{m}^3 \text{K}^4} * (2.7\text{K})^4 = 4.01 * 10^{-14} \frac{\text{J}}{\text{m}^3}
              \]
          \item
            \textbf{What is the total energy density with wavelengths between
            $1$mm and $1.01$mm? Is the Rayleigh-Jeans formula a good approximation
            at these wavelengths?}
          \end{enumerate}
  \end{enumerate}
  \chapter{Math Interlude A: Complex Numbers and Linear Operators}
\end{document}
