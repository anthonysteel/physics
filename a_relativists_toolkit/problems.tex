\documentclass[9pt]{report}
\usepackage{graphicx}
\usepackage[utf8]{inputenc}
%\usepackage[T1]{fontenc}
\usepackage{textcomp}
%\usepackage[dutch]{babel}
\usepackage{amsmath, amssymb}

\begin{document}
\title{A Relativist's Toolkit\protect\\ Problems}
\author{Anthony Steel}
\date{\today}
\maketitle
\chapter{Fundamentals}
\begin{enumerate}
  \item \textbf{The surface of a two-dimensional cone is embedded in
      three-dimensional flat space. The cone has an opening angle of
      $2\alpha$. Points on the cone which all have the same distance
      $r$ from the apex define a circle, and $\phi$ is the angle that
      runs along the circle.
    }
    \begin{enumerate}
      \item \textbf{Write down the metric of the cone, in terms of the
        coordinates $r$ and $\phi$.}


        The opening angle of the cone relates the height and the 
        raduis of the cone as:
        \[
          \begin{align}
            2\alpha &= 2\tan^{-1}(\frac{r}{h})\\
            \frac{r}{h} &= \tan\alpha
          \end{align}
        \]
      \item \textbf{Find the coordinate transformation $x(r,\phi_)$,
          $y(r,\phi)$ that brings the metric into the form $ds^2
        = dx^2+dy^2$. Do these coordinates cover the entire
        two-dimensional plane?}
      \item \textbf{Prove that any vector parallel transported along a
        circle of constant $r$ on the surface of the cone ends up rotated
        by an angle $\beta$ after a complete trip. Express $\beta$ in terms
        of $\alpha$.}
    \end{enumerate}
\end{enumerate}
\end{document}
