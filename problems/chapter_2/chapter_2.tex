\documentclass[a4paper]{article}

\usepackage[utf8]{inputenc}
%\usepackage[T1]{fontenc}
\usepackage{textcomp}
%\usepackage[dutch]{babel}
\usepackage{amsmath, amssymb}

\begin{document}
\title{Chapter 2 Problems}
\author{Anthony Steel}
\date{\today}
\maketitle
\begin{enumerate}
  \item \textbf{The metric of flat, three-dimensional Euclidean space is:} \[
      ds^2 = dx^2 + dy^2 + dz^2
  \]
  \textbf{Show that the metric components $g_{uv}$ in spherical polar
  coordinates $r, \theta, \phi$ defined by:}
  \[
    \begin{align*}
      &r = \sqrt{x^2 + y^2 + z^2} \\
      &\cos\theta = \frac{z}{r}, \\
      &\tan\phi = \frac{y}{x}
    \end{align*}
  \]
  \textbf{is given by:}
  \[
    s^2 = dr^2 + r^2 d\theta^2 + r^2 \sin^2\thetad\phi^2
  \]
 $g_{uv}$ is a tensor of type $\left( 0,2 \right) $ and therefore transforms as:
  \[
    g_{\mu',\nu'} = g_{\mu,\nu} \frac{\partial x^\mu}{\partial x^{\mu'}}\frac{\partial x^\nu}{\partial x^{\nu'}}
  \]
  (see page 22 for the general \textit{tensor transformation law}). The above
  equation uses Einstein index notation indicating that $\mu$ and $\nu$ are to
  to be summed from 1 to 3 and the free indices, $\mu'$ and $\nu'$, are
  enumerated through all possible combinations:
  \[
    \begin{matrix}
      g_{1'1'}  && g_{1'2'} && g_{1'3'} \\
      g_{2'1'} && g_{2'2'} && g_{2'3'} \\
      g_{3'1'} && g_{3'2'} && g_{3'3'}
    \end{matrix}
  \]
  Starting with:
  \[
    g_{1',1'} =
    \sum_{\mu=1}^{3} \sum_{\nu=1}^{3} g_{\mu,\nu} \frac{\partial x^\mu}{\partial x^{1'}} \frac{\partial x^\nu}{\partial x^{1'}}
  \]
  \[
    = \sum_{\mu=1}^{3}
    g_{\mu, 1} \frac{\partial x^\mu}{\partial x^{1'}} \frac{\partial x^1}{\partial x^{1'}} +
    g_{\mu, 2} \frac{\partial x^\mu}{\partial x^{1'}} \frac{\partial x^2}{\partial x^{1'}} +
    g_{\mu, 3} \frac{\partial x^\mu}{\partial x^{1'}} \frac{\partial x^3}{\partial x^{1'}}
  \]
  The off diagonal elements of the Euclidean metric written in matrix for is an
  identity matrix. This reduces the above summation from a possible 9
  components to the following three: \[
    g_{1',1'} =
    \frac{\partial x^1}{\partial x^{1'}} \frac{\partial x^1}{\partial x^{1'}} +
    \frac{\partial x^2}{\partial x^{1'}} \frac{\partial x^2}{\partial x^{1'}} +
    \frac{\partial x^3}{\partial x^{1'}} \frac{\partial x^3}{\partial x^{1'}} =
    \Bigg( \frac{\partial x^1}{\partial x^{1'}}\Bigg)^2+
    \Bigg(\frac{\partial x^2}{\partial x^{1'}}\Bigg)^2+
    \Bigg(\frac{\partial x^3}{\partial x^{1'}}\Bigg)^2
  \]
  Converting the notation to reflect the choices of basis:
  \[
    \begin{align*}
    g_{r,r} &=
    \Bigg( \frac{\partial x}{\partial r}\Bigg)^2+
    \Bigg(\frac{\partial y}{\partial r}\Bigg)^2+
    \Bigg(\frac{\partial z}{\partial r}\Bigg)^2\\
    g_{\theta,\theta} &=
    \Bigg( \frac{\partial x}{\partial \theta}\Bigg)^2+
    \Bigg(\frac{\partial y}{\partial \theta}\Bigg)^2+
    \Bigg(\frac{\partial z}{\partial \theta}\Bigg)^2\\
    g_{\phi,\phi} &=
    \Bigg( \frac{\partial x}{\partial \phi}\Bigg)^2+
    \Bigg(\frac{\partial y}{\partial \phi}\Bigg)^2+
    \Bigg(\frac{\partial z}{\partial \phi}\Bigg)^2
    \end{align*}
  \]
  First, find an equation for $x$ in terms of $r$. Rearranging:
  \[
    \begin{align*}
      &r = \sqrt{x^2 + y^2 + z^2} \to r^2= x^2+z^2+y^2,\\
      &\cos\theta = \frac{z}{r} \to z = r\cos\theta, \\
      &\tan\phi = \frac{y}{x} \to y = x\tan\phi
    \end{align*}
  \]
  Substituting the second and third equation into the first gives:
  \[
    \begin{align*}
      r^2&=x^2+(r\cos\theta)^2+(x\tan\phi)^2\\
      r^2&=x^2+r^2\cos^2\theta+x^2\tan^2\phi\\
      r^2-r^2\cos\theta&=x^2+x^2\tan^2\phi\\
      (1-\cos^2\theta)r^2&=(1+\tan^2\phi)x^2\\
      r^2\sin^2\theta &=(1+\tan^2\phi)x^2\\
      x&=r\frac{\sin\theta}{\sqrt{1+\tan^2\phi}}\\
    \end{align*}
  \]
  Differentiating:
  \[
    \begin{align*}
      \frac{\partial x}{\partial r} &= \frac{\sin\theta}{\sqrt{1+\tan^2\phi}},\\
      \frac{\partial x}{\partial \theta} &= -r\frac{\cos\theta}{\sqrt{1+\tan^2\phi}}\\
      \frac{\partial x}{\partial \phi} &=
      -r\frac{\sin\theta}{(1+\tan^2\phi)^{\frac{3}{2}}}\tan\phi\frac{1}{\cos^2\phi} \\
      &=-r\frac{\sin\theta}{(1+\tan^2\phi)^{\frac{3}{2}}}\frac{\sin\phi}{\cos^3\phi} \\
    .\end{align*}
  \]
  Second, find an equation for $y$ in terms of $r$, and then differentiate:
  \[
  \begin{align*}
    y&= x \tan\phi\\
     &= r\frac{\sin\theta}{\sqrt{1+\tan^2\phi}} \tan\phi \\
    \frac{\partial y}{\partial r}  &= \frac{\sin\theta}{\sqrt{1+\tan^2\phi}}\tan\phi\\
    \frac{\partial y}{\partial \theta}  &= -r\frac{\cos\theta}{\sqrt{1+\tan^2\phi}} \tan\phi \\
    \frac{\partial y}{\partial \phi}  &= \frac{-r\sin\theta}{(1+\tan^2\phi)^\frac{3}{2}}\frac{1}{\cos^2\phi}
  \end{align*}
  \]
  Lastly, $z$ in terms of $r$ simply follows from the definition of them
  transformation:
  \[
  z = r\cos\theta \to \frac{\partial z}{\partial r} = \cos\theta,
      \frac{\partial z}{\partial \theta} = r \sin\theta,
      \frac{\partial z}{\partial \phi} = 0
  \]
  Therefore:
  \[
    \begin{align*}
      g_{r,r}&=\frac{\sin\theta^2}{1+\tan^2\phi} + \frac{\sin^2\theta}{1+\tan^2\phi}\tan^2\phi + \cos^2\theta\\
      g_{r,r}&=\frac{\sin\theta^2 + \sin^2\theta\tan^2\phi}{1+\tan^2\phi} +\frac{(1+\tan^2\phi)\cos^2\theta}{1+\tan^2\phi}\\
      g_{r,r}&=\frac{\sin\theta^2 + \sin^2\theta\tan^2\phi+(1+\tan^2\phi)\cos^2\theta}{1+\tan^2\phi}\\
      g_{r,r}&=\frac{\sin^2\theta + \cos^2\theta+\sin^2\theta\tan^2\phi+\tan^2\phi\cos^2\theta}{1+\tan^2\phi}\\
      g_{r,r}&=\frac{1+\tan^2\phi}{1+\tan^2\phi}\\
      g_{r,r}&=1\\
    \end{align*}
  \]
  \[
    \begin{align*}
      g_{\theta,\theta}&=
      r^2 \frac{\cos^2\theta}{1+\tan^2\phi}+
      r^2 \frac{\cos^2\theta}{1+\tan^2\phi}\tan^2\phi
      + r^2\sin^2\theta\\
      g_{\theta,\theta}&=
      r^2 \frac{\cos^2\theta}{1+\tan^2\phi}+
      r^2 \frac{\cos^2\theta}{1+\tan^2\phi}\tan^2\phi
      + r^2\sin^2\theta\frac{1+\tan^2\phi}{1+\tan^2\phi}\\
      g_{\theta,\theta}&=
      r^2 \frac{\cos^2\theta+\cos^2\theta\tan^2\phi+\sin^2\theta+\tan^2\phi\sin^2\theta}{1+\tan^2\phi}\\
      g_{\theta,\theta}&=
      r^2 \frac{(\cos^2\theta+\sin^2\theta)+(\cos^2\theta+\sin^2\theta)\tan^2\phi}{1+\tan^2\phi}\\
      g_{\theta,\theta}&=
      r^2 \frac{1+\tan^2\phi}{1+\tan^2\phi}\\
      g_{\theta,\theta}&= r^2 \\
    \end{align*}
  \]
  \[
    \begin{align*}
      g_{\phi,\phi}&= r^2\sin^2\theta\Bigg(\frac{1}{(1+\tan^2\phi)^{3}\cos^4\phi} +
      \frac{\sin^2\phi}{(1+\tan^2\phi)^3\cos^6x}\Bigg)\\
      &=r^2\sin^2\theta\Bigg(\frac{\cos^2\phi+\sin^2\phi}{(1+\tan^2\phi)^3\cos^6\phi}\Bigg)\\
      &=r^2\sin^2\theta\frac{1}{\big(\frac{\sin^2\phi+\cos^2\phi}{\cos^2\phi}\big)^3\cos^6\phi}\\
      &=r^2\sin^2\theta
    \end{align*}
  \]
  Therefore the metric components are:
  \[
    g_{\mu, \nu} =
    \begin{bmatrix}
      1 && 0 && 0 \\
      0 && r^2 && 0 \\
      0 && 0 && r^2\sin^2\theta
    \end{bmatrix}
  \] 
  \end{enumerate}
\end{document}
