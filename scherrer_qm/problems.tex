\documentclass[9pt]{report}
\usepackage{graphicx}
\usepackage[utf8]{inputenc}
\usepackage{gensymb}
%\usepackage[T1]{fontenc}
\usepackage{textcomp}
%\usepackage[dutch]{babel}
\usepackage{amsmath, amssymb}

\begin{document}
\title{Scherrer's Quantum Mechanics\protect\\ Problems}
\author{Anthony Steel}
\date{\today}
\maketitle
\chapter{The Origin Of Quantum Mechanics}
  \begin{enumerate}
    \item \textbf{Assume that a human body emits blackbody radiation at the
      standard body temperature.}
      \begin{enumerate}
        \item \textbf{Estimate how much energy is radiated by the body in one hour.}

          The power emmitted by a blackbody is given by:
          \[
            P = \sigma A T^4
          \]
          where $A$ is the surface area of the body, $T$ is the temperature
          and $\sigma$ is the Stefan-Boltzmann constant. Therefore
        the energy radiated by a body in a given time interval $\Delta t$:
          \[
            E = \sigma A T^4 \Delta t
          \]
          The surface area of the human body is approximately $2\text{m}^2$,
          the average body temperature is $36.1^{\circ}\text{C} = 309.25 \text{K}$,
          and there are $3600$s in an hour. Therefore:
          \[
            \begin{align}
            E_\text{hour}
            &= \Big(5.67 * 10^{-8} \frac{\text{J}}{\text{s m}^2\text{K}^4}\Big)( 2\text{m}^2 )(309.25\text{K})^4(3600\text{s})\\
            &= 3733 \text{kJ}
            \end{align}
          \]

        \item \textbf{At what wavelength does this radiation reach a maximum}

          The formula for the maximum wavelength is:
          \[
            \lambda_\text{peak} = \frac{w}{T}
          \]
          where $w = 2.90 * 10^{-3} \text{m K}$ and $T = 309.25\text{K}$ as
          before. Therefore the maximum wavelength is:
          \[
            \lambda_\text{peak} = \frac{2.9 * 10^{-3} \text{m K}}{309.25\text{K}} = 9.37 * 10^{-6} \text{m}
          \]
      \end{enumerate}
        \item \textbf{A distant red star is observed to have a blackbody spectrum with a
          maximum at a wavelength of $3500$\AA [$1$\AA = $10^{-10}$ m]. What is the
          temperature of the star?}

          Inverting the formula from the pervious question:
          \[
            T = \frac{w}{\lambda_\text{peak}}
          \]
          giving:
          \[
            T = \frac{2.9 * 10^{-3} \text{m K}}{ 3500 * 10^{-10} \text{m}} = 51428 \text{K}
          \]

        \item \textbf{The universe is filled with blackbody radiation at a temperature
            of $2.7$K left over from the Big Bang. [This radiation was disvoered in
          1965 by Bell Laboratory scientists who thought at one point that they were seeing
          interference from pigeon droppings on their microwave reciever.}
          \begin{enumerate}
          \item
            \textbf{What is the total energy density of this radiation?}

            The  energy density of the radiation is given by:
              \[
                \rho = a T^4
              \]
              where $a = 7.56 * 10^{-16} \frac{\text{J}}{\text{m}^3 \text{K}^4}$.
              Therefore:
              \[
                \rho = 7.56 * 10^{-16} \frac{\text{J}}{\text{m}^3 \text{K}^4} * (2.7\text{K})^4 = 4.01 * 10^{-14} \frac{\text{J}}{\text{m}^3}
              \]
          \item
            \textbf{What is the total energy density with wavelengths between
            $1$mm and $1.01$mm? Is the Rayleigh-Jeans formula a good approximation
            at these wavelengths?}
          \end{enumerate}
  \end{enumerate}
  \chapter{Math Interlude A: Complex Numbers and Linear Operators}
  \begin{enumerate}
    \item \textbf{Eavluate all of th efollowing and express all of your final
      answers in the form $a + bi$:}
        \begin{enumerate}
          \item $i (2-3i)(3 + 5i)$

            \[
              \begin{align}
                i (2-3i)(3 + 5i) &= i (6 + 10i - 9i + 15) \\
                                 &= i (21 + i)\\
                                 &= -1 + 21i
              \end{align}
            \]

      \item $ i/i-1 $

      \[
        \begin{align}
            i/i-1 &= e^{i\pi/2} / (\sqrt{2} e^{3i\pi/4})\\
                  &= \frac{1}{\sqrt{2}} e^{-i\pi/4}
        \end{align}
      \]

    \item $(1 + i)^{30}$

      \[
      \begin{align}
        (1 + i)^{30} &= (\sqrt{2} e^{i\pi/4})^{30}\\
                     &= 2^{15} e^{15i\pi/2}\\
                     &= 2^{15} e^{3i\pi/2} \\
                     &= -2^{15} i\\
      \end{align}
    \]
        \end{enumerate}
      \item
      \item
      \item \textbf{Suppose that a complex number $z$ has the property that $z^* = z$. What does this indicate about $z$?}

        This indicates that $z$ is a real number.

      \item \textbf{Reduce $i^i$ to a real number}
        \[
          i^i = (e^{i\pi/2})^i = e^{-\pi/2}

        \]
      \item \textbf{What is wrong with the following argument?}

        \[
          \begin{align}
          \sqrt{\frac{1}{-1}} &= \frac{\sqrt{1}}{\sqrt{-1} } \\
          \sqrt{-1} &= \frac{1}{i} \\
          i &= \frac{1}{i} \\
          (i)(i) &= 1 \\
          -1 &= 1\\
        \end{align}
        \]

        The first line is a false equivalence.
        \[
          \sqrt{\frac{1}{-1}} = \sqrt{-1} = i = e^{i\pi/2}
        \]
        and
        \[
          \frac{\sqrt{1}}{\sqrt{-1} }= \frac{1}{\sqrt{-1}} = e^{-i\pi/2}
        \]
      \item
        \textbf{Determine which of the following are linear operators, and which
        are not.}
        \begin{enumerate}
          \item \textbf{The parity operator $\Pi[f(x)]=f(-x)$.}
            \[
              \begin{align}
                \Pi[f(x) + g(x)] &= f(-x) + g(-x)\\
                                 &= \Pi[f(x)] + \Pi[g(x)]
              \end{align}
            \]
            \[
              \begin{align}
                \Pi[cf(x)] &= cf(-x) \\
                           &= c\Pi[f(x)]
              \end{align}
            \]
            Therefore the parity operator is linear.
          \item \textbf{The transformation operator $T[f(x)] = f(x+1)$.}
            \[
              \begin{align}
                T[f(x) + g(x)] &= f(x+1) + g(x+1)\\
                               &= T[f(x)] + T[g(x)]
              \end{align}
            \]
            \[
              \begin{align}
                T[cf(x)] &= cf(x+1)\\
                         &= cT[f(x)]
              \end{align}
            \]
            Therefore the transformation operator is linear.
          \item \textbf{The operator $L[f(x)] = f(x) + 1$}
            \[
              L[f(x) + g(x)] = f(x) + g(x) + 1 \neq L[f(x)] + L[g(x)]
            \]
            \[
              L[cf(x)] = cf(x) + 1 \neq cL[f(x)]
            \]
            Therefore the operator is not linear.
        \end{enumerate}
      \item \textbf{Consider the identity operator $I$, defined by $I[f(x)] = f(x)$.}
        \begin{enumerate}
          \item \textbf{Show that $I$ is a linear operator.}
            \[
              I[f(x) + g(x)] = f(x) + g(x) = I[f(x)] + I[g(x)]
            \]
            \[
              I[cf(x)] = cf(x) = cI[f(x)]
            \]
            Therefore the identity operator is linear.
          \item \textbf{Find the eigenfunctions and corresponding eigenvalues of $I$}

            The eigenfunctions are given by:
            \[
              I[f(x)] = f(x) = cf(x)
            \]
            Therefore every function is an eigen function of the identity operator
            with eigenvalue $c=1$.
        \end{enumerate}
      \item \textbf{Suppose that the function $f(x)$ is an eigenfunction of the linear
          operator $P$ with eigenvalue $p$, and $f(x)$ is also an eigenfunction of the
          linear operator $Q$ with eigenvalue $q$. Show that $PQ[f(x)] = QP[f(x)]$, where
          $PQ[f(x)]$ means to first apply the operator $Q$ to $f(x)$, and then apply
        $P$ to the result.}

        \[
          \begin{align}
            PQ[f(x)] &= P[Q[f(x)] \\
                     &= P[q f(x) ] \\
                     &= pqf(x)
          \end{align}
        \]
        \[
          \begin{align}
            QP[f(x)] &= Q[P[f(x)] \\
                     &= Q[p f(x) ] \\
                     &= qpf(x)
          \end{align}
        \]
        Because the eigenvalues are real numbers they comute. Thefore:
        \[
          PQ[f(x)] = QP[f(x)]
        \]

      \item \textbf{Consider the square of the derivative operator}
        \begin{enumerate}
          \item
            \textbf{Show that $D^2$ is a linear operator}
            \[
              \begin{align}
                D^2[f(x) + g(x)] &= \frac{d^2}{dx^2} \Big( f(x) + g(x) \Big)\\
                                 &= \frac{d^2f}{dx^2} + \frac{d^2g}{dx^2}\\
                                 &= D^2[f(x)] + D^2[g(x)]
              \end{align}
            \]
            \[
              \begin{align}
                D^2[cf(x)] &= \frac{d^2}{dx^2}\Big( cf(x)\Big) \\
                           &= c \frac{d^2f}{dx^2}\\
                           &= c D^2[f(x)]
              \end{align}
            \]
            Therfore $D^2$ is a linear operator.
          \item
            The eigenfunctions of $D^2$ are given by:
            \[
              D^2[f(x)] = \frac{d^2f}{dx^2} = cf(x)
            \]
            which is equivalent to solving the homogenous linear second-order
            differential equation:
            \[
              \frac{d^2f}{dx^2} - cf(x) =0
            \]
            This equation has the solution:
            \[
              f(x) = Ae^{i\sqrt{c}x} + Be^{-i\sqrt{c}x}
            \]
          \item
            \textbf{Give an example of an eigenfunction of $D^2$ which is not
            an eigenfunction of $D$}
            \[
              f(x) = A\cos(\sqrt{c} x)
            \]
        \end{enumerate}
      \item \textbf{Let $f(x)$ be an eigenfunction of a linear operator $L$ with
          eigenvalue $a$. Show that $cf(x)$ (where $c$ is a constant) is an
          eigenfunction of $L$ with eigenvalue $a$.}
  \end{enumerate}
\end{document}
