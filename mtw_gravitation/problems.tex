\documentclass[9pt]{report}
\usepackage{graphicx}
\usepackage[utf8]{inputenc}
%\usepackage[T1]{fontenc}
\usepackage{textcomp}
%\usepackage[dutch]{babel}
\usepackage{amsmath, amssymb}

\begin{document}
\title{Misner, Thorne and Wheeler's Gravitation\protect\\ Problems}
\author{Anthony Steel}
\date{\today}
\maketitle
\chapter{}
\chapter{}
\chapter{The Electromagnetic Field}
\begin{enumerate}
  \item \textbf{Derive equations:}
    \begin{equation}
       \mid  \mid F^\alpha_\beta  \mid \mid =
       \begin{Vmatrix}
         0 & E_x & E_y & E_z\\
         E_x & 0 & B_z & -B_y\\
         E_y & -B_z & 0 & B_x\\
         E_z & B_y & -B_x & 0\\
       \end{Vmatrix}
       \label{mixed-faraday}
     \end{equation}
    \textbf{and}
    \begin{equation}
       \mid  \mid F_\alpha_\beta  \mid \mid =
       \begin{Vmatrix}
         0 & -E_x & -E_y & -E_z\\
         E_x & 0 & B_z & -B_y\\
         E_y & -B_z & 0 & B_x\\
         E_z & B_y & -B_x & 0\\
       \end{Vmatrix}
     \end{equation}
    \textbf{for the components of Faraday by comparing:}
    \begin{equation}
      dp^\alpha / d\tau =  e F^\alpha_\beta u^\beta \label{momentum}
    \end{equation}
    \textbf{with}
    \begin{equation}
      \frac{d\textbf{p}}{d\tau} = \frac{1}{\sqrt{1-\textbf{v}^2}} \frac{d\textbf{p}}{dt} = \frac{e}{\sqrt{1-\textbf{v}^2} }(\textbf{E} + \textbf{v}\times\textbf{B}) = e(u^0\textbf{E} + \textbf{u} \times \textbf{B}) \label{momentum_spatial}
    \end{equation}
    \begin{equation}
      \frac{dp^0}{d\tau} = \frac{1}{\sqrt{1-\textbf{v}^2}} \frac{dE}{dt} = \frac{1}{\sqrt{1-\textbf{v}^2}}e\textbf{E} \cdot \textbf{v} = e \textbf{E} \cdot \textbf{u} \label{momentum_time}
    \end{equation}
    \textbf{and by using definition:}
    \begin{equation}
      F_\alpha_\beta = \eta_\alpha_\gamma F^\gamma_\beta \label{metric}
    \end{equation}
    Consider equation \ref{momentum} for the index $\alpha=0$:
    \[
      \frac{dp^0}{d\tau} = e [ F^0_0 u^0 + F^0_1 u^1 + F^0_2 u^2 + F^0_3 u^3 ]
    \]
    Equate this with \ref{momentum_time}:
    \[
      e [ F^0_0 u^0 + F^0_1 u^1 + F^0_2 u^2 + F^0_3 u^3 ] = e \textbf{E} \cdot \textbf{u} = e [E_1u^1 + E_2 u^2 + E_3 u^3]
    \]
    It is clear that:
    \[
      \begin{align}
      F_0^0 &= 0 \\
      F^0_1u^1 &= E_1 u^1 \Rightarrow F^0_1 = E_1\\
      F^0_2u^2 &= E_2 u^2 \Rightarrow F^0_2 = E_2\\
      F^0_3u^3 &= E_3 u^3 \Rightarrow F^0_3 = E_3\\
      \end{align}
    \]
    Now equating equation \ref{momentum_spatial} with the remaining components
    of equation \ref{momentum}:
    \[
      \frac{dp^1}{d\tau} = e [ F^1_0 u^0 + F^1_1 u^1 + F^1_2 u^2 + F^1_3 u^3 ] = e [E_1 u^0 + B_3 u^2 - B_2 u^3 ]
    \]
    \[
      \frac{dp^2}{d\tau} = e [ F^2_0 u^0 + F^2_1 u^1 + F^2_2 u^2 + F^2_3 u^3 ] = e [E_2 u^0 + B_1 u^3 - B_3 u^2 ]
    \]
    \[
      \frac{dp^3}{d\tau} = e [ F^3_0 u^0 + F^3_1 u^1 + F^3_2 u^2 + F^3_3 u^3 ] = e [E_3 u^0  + B_2 u^1 - B_1 u^2 ]
    \]
    and equating components as before:
    \[
    \begin{align}
      F^1_0 u^0 &= E_1 u^0 \Rightarrow F_0^1 = E_1 \\
      F^1_1 u^1 &= 0 \Rightarrow F_1^1 = 0 \\
      F^1_2 u^2 &= B_3 u^2 \Rightarrow F_2^1 = B_3 \\
      F^1_3 u^3 &= -B_2 u^3 \Rightarrow F_3^1 = -B_2 \\
      F^2_0 u^0 &= E_2 u^0 \Rightarrow F^2_0 = E_2 \\
      F^2_1 u^1 &= -B_3 u^1 \Rightarrow F^2_1 = - B_3 \\
      F^2_2 u^2 &= 0 \Rightarrow F^2_2 = 0 \\
      F^2_3 u^3 &= B_1 u^3 \Rightarrow F^2_3 = B_1 \\
      F^3_0 u^0 &= E_3 u^0 \Rightarrow F^3_0 = E_3 \\
      F^3_1 u^1 &= B_2 u^1 \Rightarrow F^3_1 = B_2 \\
      F^3_2 u^2 &= -B_1 u^2 \Rightarrow F^3_2 = -B_1 \\
      F^3_3 u^3 &= 0 \Rightarrow F^3_3 = 0
    \end{align}
    \]
    Collecting all of these components in matrix form and relabling indices with
    the following mapping:
    \[
      \begin{align}
        \alpha = 1 &\rightarrow x \\
        \alpha = 2 &\rightarrow y \\
        \alpha = 3 &\rightarrow z \\
      \end{align}
    \]
    gives equation \ref{mixed-faraday}:
    \[
       \mid  \mid F^\alpha_\beta  \mid \mid =
       \begin{Vmatrix}
         F^0_0 & F^0_1 & F^0_2 & F^0_3\\
         F^1_0 & F^1_1 & F^1_2 & F^1_3\\
         F^2_0 & F^2_1 & F^2_2 & F^2_3\\
         F^3_0 & F^3_1 & F^3_2 & F^3_3\\
       \end{Vmatrix}
       =
       \begin{Vmatrix}
         0 & E_x & E_y & E_z\\
         E_x & 0 & B_z & -B_y\\
         E_y & -B_z & 0 & B_x\\
         E_z & B_y & -B_x & 0\\
       \end{Vmatrix}
    \]
    Now equation \ref{metric} can be used to convert the mixed Faraday tensor
    to the fully covariant one. Remeber that for all components $\alpha \neq \beta$
    the Minkowski metric is zero. Therefore the only non-zero components in the
    sums created by the summation convention in equation \ref{metric} are:
    \[
      \begin{align}
        F_0_0 &= \eta_0_0 F_0^0 \Rightarrow F_{00} = -F^0_0 \\
        F_0_1 &= \eta_0_0 F_1^0 \Rightarrow F_{01} = -F^0_1\\
        F_0_2 &= \eta_0_0 F_2^0 \Rightarrow F_{02} = -F^0_2\\
        F_0_3 &= \eta_0_0 F_3^0 \Rightarrow F_{03} = -F^0_3\\
        F_1_0 &= \eta_1_1 F_0^1 \Rightarrow F_{10} = F_0^1\\
        F_1_1 &= \eta_1_1 F_1^1 \Rightarrow F_{11} = F_1^1\\
        F_1_2 &= \eta_1_1 F_2^1 \Rightarrow F_{12} = F_2^1 \\
        F_1_3 &= \eta_1_1 F_3^1 \Rightarrow F_{13} = F_3^1 \\
        F_2_0 &= \eta_2_2 F_0^2 \Rightarrow F_0_2 = F_0^2 \\
        F_2_1 &= \eta_2_2 F_1^2 \Rightarrow F_1_2 = F_1^2\\
        F_2_2 &= \eta_2_2 F_2^2 \Rightarrow F_2_2 = F_2^2\\
        F_2_3 &= \eta_2_2 F_3^2 \Rightarrow F_2_3 = F_2^3 \\
        F_3_0 &= \eta_3_3 F_0^3 \Rightarrow F_3_0 = F_0^3\\
        F_3_1 &= \eta_3_3 F_1^3 \Rightarrow F_3_1 = F_1^3\\
        F_3_2 &= \eta_3_3 F_2^3 \Rightarrow F_3_2 = F_2^3\\
        F_3_3 &= \eta_3_3 F_3^3 \Rightarrow F_3_3 = F_3^3\\
      \end{align}
    \]
    Collecting the components into matrix form recovers the fully covariant
    Faraday tensor:
    \begin{equation}
       \mid  \mid F_\alpha_\beta  \mid \mid =
       \begin{Vmatrix}
         0 & -E_x & -E_y & -E_z\\
         E_x & 0 & B_z & -B_y\\
         E_y & -B_z & 0 & B_x\\
         E_z & B_y & -B_x & 0\\
       \end{Vmatrix}
     \end{equation}
\item \textbf{From the transformation laws for components of vectors and 1-forms,
  derive the transformation law:}
  \[
    S^{\mu'}^{\nu'}_{\lambda'} = S^\alpha^\beta_\gamma \Lambda^{\mu'}_\alpha \Lambda^{\nu'}_\beta \Lambda^\gamma_{\lambda'}
  \]
  Consider the tensor $\textbf{S}$ of rank $(2, 1)$, in geometric notation the transformation between
  two sets basis vectors and 1-forms reads:
  \[
  \textbf{S}(\bf{\sigma}, \bf{\rho}, \bf{\nu}) =
  \textbf{S}(\bf{\sigma'}, \bf{\rho'}, \bf{\nu'})
  \]
  In component form this reads:
  \begin{equation}
    S^\alpha^\beta_\gamma \sigma_\alpha \rho_\beta \nu^\gamma = S^{\mu'}^{\nu'}_{\lambda'} \sigma_{\mu'} \rho_{\nu'} \nu^{\lambda'} \label{2-1_tensor_transformation}
  \end{equation}
  Using the Lorentz transformation laws to transform one basis into the other
  for $\bf{\sigma}$, $\bf{\rho}$, $\bf{\nu}$ gives:
  \[
  \begin{align}
    \sigma_\alpha &= \Lambda^{\mu'}_{\alpha} \sigma_{\mu'} \\
    \rho_\beta &= \Lambda^{\nu'}_{\beta} \rho_{\mu'} \\
    \nu^\beta &= \Lambda^{\gamma}_{\lambda'} \nu^{\lambda'} \\
  \end{align}
  \]
  and substituting these transformations into equation \ref{2-1_tensor_transformation}:
  \[
    S^{\mu'}^{\nu'}_{\lambda'} \sigma_{\mu'} \rho_{\nu'} \nu^{\lambda'} = S^\alpha^\beta_\gamma (\Lambda^{\mu'}_{\alpha} \sigma_{\mu'}) (\Lambda^{\nu'}_{\beta} \rho_{\mu'}) (\Lambda^{\gamma}_{\lambda'} \nu^{\lambda'})
  \]
  \[
    S^{\mu'}^{\nu'}_{\lambda'} \sigma_{\mu'} \rho_{\nu'} \nu^{\lambda'} = S^\alpha^\beta_\gamma \Lambda^{\mu'}_{\alpha} \Lambda^{\nu'}_{\beta} \Lambda^{\gamma}_{\lambda'} \sigma_{\mu'} \rho_{\mu'} \nu^{\lambda'}
  \]
  Equating the components gives the desired transformation law:
  \[
    S^{\mu'}^{\nu'}_{\lambda'} = S^\alpha^\beta_\gamma \Lambda^{\mu'}_{\alpha} \Lambda^{\nu'}_{\beta} \Lambda^{\gamma}_{\lambda'}
  \]
\item  \textbf{Raising and lowering indices. Derive:}
  \begin{equation}
    S^\alpha_\beta_\gamma = \eta_\beta_\mu S^\alpha^\mu_\gamma
  \end{equation}
  \textbf{and:}
  \begin{equation}
    S^\alpha^\mu_\gamma = \eta^\mu^\beta S^\alpha_\beta_\gamma
  \end{equation}
  \textbf{from:}
\item
\item
\item
\item
  \textbf{Maxwell's Equations. Show, by explicit examination of components, that
  the geometric laws}
  \begin{equation}
    F_{\alpha\beta,\gamma} + F_{\beta\gamma,\alpha} + F_{\gamma\alpha, \beta} = 0
  \end{equation}
  \begin{equation}
    F^{\alpha\beta}_{,\beta} = 4\pi J^\alpha
  \end{equation}
  \textbf{do reduce to Maxwell's equations}
  \begin{equation}
    \nabla \cdot \textbf{B} = 0
  \end{equation}
  \begin{equation}
    \frac{\partial \textbf{B}}{\partial t} + \nabla \times \textbf{E} = 0
  \end{equation}
  \begin{equation}
    \nabla \cdot \textbf{E} = 4\pi \bf{\rho}
  \end{equation}
  \begin{equation}
    \frac{\partial \textbf{E}}{\partial t} - \nabla \times \textbf{B} = -4\pi \textbf{J}
  \end{equation}
\item
\item
\item \textbf{More differentiation. (a) Justify the formula,}
  \[
    d(u^\mu u_\mu) / d\tau = 2 u_\mu (d u^\mu / d\tau),
  \]
  \textbf{by writing out the summation $u^\mu u_\mu = \eta_\mu_\nu u^\mu u^\nu$ explicitly}
  Writing out the components explicitly yields:
  \[
    \begin{align}
    u^\mu u_\mu &= \eta_{00} u^0 u^0 + \eta_{11} u^1 u^1 + \eta_{22} u^2 u^2 + \eta_{33} u^3 u^3\\
                &= \eta_{00} (u^0)^2 + \eta_{11} (u^1)^2 + \eta_{22} (u^2)^2 + \eta_{33} (u^3)^2 \\
                &= - (u^0)^2 + (u^1)^2 + (u^2)^2 + (u^3)^2
  \end{align}
  \]
  Taking a total derivative of the above with respect to $\tau$:
  \[
    \begin{align}
    \frac{d}{d\tau} (u^\mu u_\mu) &= \frac{d}{d\tau} (-(u^0)^2 + (u^1)^2 + (u^2)^2 + (u^3)^2) \\
                                  &= -2 u^0 \frac{du^0}{d\tau} + 2 u^1 \frac{du^1}{d\tau} + 2 u^2 \frac{du^2}{d\tau}+ 2 u^3 \frac{du^3}{d\tau}\\
                                  &= 2 [ -u^0 \frac{du^0}{d\tau} + u^1 \frac{du^1}{d\tau} + u^2 \frac{du^2}{d\tau}+ u^3 \frac{du^3}{d\tau} ] \\
                                  &= 2 \eta_\mu_\nu u^\mu \frac{du^\nu}{d\tau}\\
                                  &= 2 u_\mu \frac{du^\nu}{d\tau}\\
    \end{align}
  \]
  Therefore the formula is justified.

  \textbf{b) Let $\delta$ indicate a variation or small change, and justify the
  formula:}
  \[
    \delta(F_\alpha_\beta F^\alpha^\beta) = 2 F_\alpha_\beta \delta F^\alpha^\beta
  \]
  The variation  will obey the product rule as follows and rembering that the variation of
  the metric components would be zero:
  \[
    \begin{align}
    \delta(F_\alpha_\beta F^\alpha^\beta) &= (\delta F_\alpha_\beta) F^\alpha^\beta + F_\alpha_\beta (\delta F^\alpha^\beta) \\
                                          &= (\delta (\eta_\alpha_\gamma \eta_\beta_\nu F^\alpha^\beta ) F^\gamma^\nu) + F_\alpha_\beta (\delta F^\alpha^\beta) \\
                                          &= (\delta (\eta_\alpha_\gamma) \eta_\beta_\nu F^\alpha^\beta + \eta_\alpha_\gamma \delta(\eta_\beta_\nu) F^\alpha^\beta + \eta_\alpha_\gamma \eta_\beta_\nu \delta F^\alpha^\beta  ) F^\gamma^\nu) + F_\alpha_\beta (\delta F^\alpha^\beta) \\
                                          &= \eta_\alpha_\gamma \eta_\beta_\nu \delta F^\alpha^\beta  F^\gamma^\nu + F_\alpha_\beta (\delta F^\alpha^\beta) \\
                                          &= \delta F^\alpha^\beta (\eta_\alpha_\gamma \eta_\beta_\nu  F^\gamma^\nu) + F_\alpha_\beta (\delta F^\alpha^\beta) \\
                                          &= \delta F^\alpha^\beta F_\alpha_\beta + F_\alpha_\beta \delta F^\alpha^\beta \\
                                          &= 2 F_\alpha_\beta \delta F^\alpha^\beta \\
  \end{align}
  \]
  Therefore the formula is justified.

  \textbf{c) Compute $(F_\alpha_\beta F^\alpha^\beta)_{,\mu} = ?$}
  \[
    \begin{align}
    (F_\alpha_\beta F^\alpha^\beta)_{,\mu} &= F_\alpha_\beta_{,\mu} F^\alpha^\beta + F_\alpha_\beta F^\alpha^\beta_{,\mu} \\
                                           &= (\eta_\alpha_\gamma \eta_\beta_\nu F^\alpha^\beta)_{,\mu} F^\gamma^\nu + F_\alpha_\beta F^\alpha^\beta_{,\mu} \\
                                           &= F^\alpha^\beta_{,\mu} F_\alpha_\beta + F_\alpha_\beta F^\alpha^\beta_{,\mu} \\
                                           &= 2F_\alpha_\beta F^\alpha^\beta_{,\mu} \\
    \end{align}
  \]
\end{enumerate}
\chapter{Electromagnetism and Differential Forms}
\chapter{Stress-Energy Tensor and Conservatoin Laws}
\chapter{Accelerated Observers}
\begin{enumerate}
\item \textbf{A TRIP TO THE GALACTIC NUCLEUS}

\textbf{Compute the proper time required
for the occupants of a rocket schip to travel the $\approx 30,000$ light-years
to get from the Earth to the center of the Galaxy. Asusme that they maintain
an acceleration of one earth gravity ($10^3$ cm/sec$^2$) for half the trip,
and then decelerate at one earth gravity for the remaining of the half.}

Qualitatively, the worldline of the traveller is pictured in the Figure \ref{galactic_center}.
The travellers worldine is composed of two arcs of hyperbola $AC$ and $CB$.
\begin{figure}
  \begin{center}
  \includegraphics[width=0.50\textwidth]{images/galactic_center_minkowski.jpg}
  \end{center}
  \caption{The worldline of the traveller is composed of two arcs of hyperbola
  $AC$ and $CB$. $G$ indicates the distance to the galactic center.}
  \label{galactic_center}
\end{figure}
It is clear geometrically that the total time required for the trip to the
center will be double the time to the middle of galaxy because of the symmetry
of accelerations/decelerations. The benefit of handling the situation this
way is we can use the equations derived for a positive acceleration.


As stated:
\begin{equation}
    t = g^{-1} \sinh g \tau
\end{equation}
\begin{equation}
    x = g^{-1} \cosh g \tau \label{a_dis}
\end{equation}
describe the worldline of an accelerated observer in the reference frame of an inertial observer
with respect to $x^1$, the direction of acceleration. Invert Equation \ref{a_dis}:
\[
  \tau = g^{-1} \cosh^{-1} gx
\]
Reinserting dimensional units gives:
\[
  \tau = c g^{-1} \cosh^{-1} gxc^{-2}
\]
Substituting $x=15000$ light-years gives $\tau \approx 12.25$ years for half the
trip, and $\tau \approx 24.5$ years for the full trip.

\item \textbf{ROCKET PAYLOAD}

\textbf{What fraction of the initial mass of the rocket can be payload for the
journey considered in exercise 1? Assume an ideal rocket that converts rest
mass into radiation and ejects the radiation out the back of the rocket with
100 per cent efficiency and perfect collimation.}

Consider the 4-momentum of the rocket:
\[
  mu^\alpha = m(u^t, u^x)
\]
$u^x$ is the momentum in the direction of travel and $u^t$ is mass-energy of
the rocket. The change in the mass-energy is the radiation ejected from
the rocket, and it is equal to the negative of the change in momentum of the
rocket. A decrease in mass-energy results in an increase of momentum in the
$x$ direction and vice-versa. Therefore:
\[
  \begin{align}
    \frac{d}{d\tau}(mu^t) &= -\frac{d}{d\tau}(m u^x) \\
    \frac{dm}{d\tau} u^t + m \frac{du^t}{d\tau}&= - (\frac{dm}{d\tau} u^x + m \frac{du^x}{d\tau}) \\
    (u^t + u^x)\frac{dm}{d\tau} &= - m (\frac{du^t}{d\tau}+ \frac{du^x}{d\tau}) \\
    (u^t + u^x)\frac{dm}{d\tau} &= - m \frac{d}{d\tau}(u^t+u^x) \\
    \frac{1}{m}\frac{dm}{d\tau} &=  \frac{1}{(u^t + u^x)}\frac{d}{d\tau}(u^t+u^x) \\
  \end{align}
\]
This equation can be integrated as:
\[
  \begin{align}
    \ln\frac{m}{m_0} &= - \ln(u^t+u^x) \\
    \frac{m}{m_0} &= \frac{1}{u^t+u^x} \\
    m &= \frac{1}{u^t+u^x}m_0 \\
  \end{align}
\]
Remembering that:
\[
  u^t = \frac{dt}{d\tau} = \frac{d}{d\tau} (g^{-1} \sinh g\tau) = \cosh g\tau
\]
\[
  u^x = \frac{dx}{d\tau} = \frac{d}{d\tau} (g^{-1} \cosh g\tau) = \sinh g\tau
\]
Substituting and symplifying:
\[
  \begin{align}
    m &= \frac{1}{\cosh g\tau +\sinh g\tau }m_0 \\
      &= \frac{2}{e^{g \tau} + e^{-g \tau} + e^{g \tau} - e^{-g \tau}} m_0\\
      &= \frac{2}{2 e^{g\tau}} m_0\\
      &= e^{-g\tau} m_0\\
  \end{align}
\]
which for $\tau \approx 12.25$ years times 2 gives:
\[
  m = 6.5 * 10^{-6}m_0
\]
\item \textbf{TWIN PARADOX}

  \textbf{(a) Show that, of all timelike word lines connecting two events $A$ and
  $B$, the one with the longest lapse of proper time is the unacccelerated one.
  Hint: perform the calculation in the inertial frame of the unaccelerated
  world line.}

  Consider the proper time defined by:
  \[
    \begin{align}
      \tau &= \int_A^B \sqrt{-\eta_\mu_\nu \frac{dx^\mu}{d\lambda} \frac{dx^\nu}{d\lambda}} d \lambda\\
    \end{align}
  \]
  The problem amounts to varying the proper time and showing that :
  \[
    \delta \tau = 0
  \]
  implies no 4-acceleration (i.e. $d^2x^\nu/d\tau^2 = 0$). This problem should be familiar from the calculus of
  variations. The condition for $\delta \tau = 0$ is that $L$ must satisfy
  Lagrange's equations:
  \[
    \frac{d}{d\lambda} \frac{\partial L}{\partial \dot{x}^\gamma} = \frac{\partial L}{\partial x^\gamma}
  \]
  where $\dot{x}^\gamma = dx^\gamma/d\lambda$ and :
  \[
    L = \sqrt{-\eta_\mu_\nu \frac{dx^\mu}{d\lambda} \frac{dx^\nu}{d\lambda}}
  \]
  Therefore:
  \[
  \begin{align}
    \frac{\partial L}{\partial x^\gamma} &= -\frac{1}{2L}\frac{\partial}{\partial x^\gamma}\Big(\eta_{\mu\nu} \frac{dx^\mu}{d\lambda} \frac{dx^\nu}{d\lambda} \Big) \\
                                         &=-\frac{1}{2L}\Big(\frac{\partial\eta_{\mu\nu}}{\partial x^\gamma} \frac{dx^\mu}{d\lambda} \frac{dx^\nu}{d\lambda} + 2\eta_{\mu\nu} \frac{\partial}{\partial x^\gamma} \Big( \frac{dx^\mu}{ d\lambda} \Big) \frac{dx^\nu}{d\lambda} \Big)\\
                                         &=-\frac{1}{2L}\Big(\frac{\partial\eta_{\mu\nu}}{\partial x^\gamma} \frac{dx^\mu}{d\lambda} \frac{dx^\nu}{d\lambda} + 2\eta_{\mu\nu} \frac{d}{d\lambda} \Big( \frac{\partial x^\mu}{ \partial x^\gamma} \Big) \frac{dx^\nu}{d\lambda} \Big)\\
                                         &=-\frac{1}{2L}\Big(\frac{\partial\eta_{\mu\nu}}{\partial x^\gamma} \frac{dx^\mu}{d\lambda} \frac{dx^\nu}{d\lambda} + 2\eta_{\mu\nu} \frac{d}{d\lambda} \delta^\nu_\gamma \frac{dx^\nu}{d\lambda} \Big)\\
                                         &=-\frac{1}{2L}\frac{\partial\eta_{\mu\nu}}{\partial x^\gamma} \frac{dx^\mu}{d\lambda} \frac{dx^\nu}{d\lambda} \\
                                         &= 0
  \end{align}
  \]
  \[
    \begin{align}
      \frac{d}{d\lambda} \frac{\partial L}{\partial \dot{x}^\gamma} &= \frac{d}{d\lambda} \Big( -\frac{1}{2L} \frac{\partial}{\partial \dot{x}^\gamma}\Big(\eta_{\mu\nu} \frac{dx^\mu}{d\lambda} \frac{dx^\nu}{d\lambda} \Big) \Big)\\
                                                                    &= \frac{d}{d\lambda} \Big( -\frac{1}{2L} \Big(\frac{\partial\eta_{\mu\nu}}{\partial \dot{x}^\gamma}\frac{dx^\mu}{d\lambda} \frac{dx^\nu}{d\lambda} + \eta_{\mu\nu} \frac{\partial\dot{x}^\mu}{\partial\dot{x}^\gamma} \dot{x}^\nu + \eta_{\mu\nu} \dot{x}^\mu \frac{\partial\dot{x}^\nu}{\partial\dot{x}^\gamma} \Big) \Big) \\
                                                                    &= \frac{d}{d\lambda} \Big( -\frac{1}{2L} \Big(\frac{\partial\eta_{\mu\nu}}{\partial \dot{x}^\gamma}\frac{dx^\mu}{d\lambda} \frac{dx^\nu}{d\lambda} + \eta_{\mu\nu} \delta^\mu_\gamma \dot{x}^\nu + \eta_{\mu\nu} \dot{x}^\mu \delta^\nu_\gamma \Big) \Big) \\
                                                                    &= \frac{d}{d\lambda} \Big( -\frac{1}{2L} \Big(\frac{\partial\eta_{\mu\nu}}{\partial \dot{x}^\gamma}\frac{dx^\mu}{d\lambda} \frac{dx^\nu}{d\lambda} + 2 \eta_{\gamma\nu} \dot{x}^\nu \Big)\Big)\\
                                                                    &= - \frac{d}{d\lambda} \Big(\frac{1}{L}\eta_{\gamma\nu} \dot{x}^\nu \Big)
    \end{align}
  \]
  Now from the equation for $\tau$ it is clear that: 
  \[
    \frac{d\tau}{d\lambda} = L
  \]
  and therefore for a function $f(\tau(\lambda))$:
  \[
    \frac{df}{d\lambda} = \frac{df}{d\tau} \frac{d\tau}{d\lambda} = L \frac{df}{d\tau}
  \]
  We can use this relationship to exchange differentiation with respect to $\lambda$
  for differentiation with respect to $\tau$, such that:
  \[
    \begin{align}
      \frac{d}{d\lambda} \frac{\partial L}{\partial \dot{x}^\gamma} &= - \frac{d}{d\lambda} \Big(\frac{1}{L}\eta_{\gamma\nu} \frac{dx^\nu}{d\lambda} \Big) \\
                                                                    &= - L\frac{d}{d\tau} \Big(\frac{1}{L}\eta_{\gamma\nu} \Big(L\frac{dx^\nu}{d\tau} \Big)\Big) \\
                                                                    &= - L\eta_{\gamma\nu} \frac{d^2x^\nu}{d\tau^2}
    \end{align}
  \]
  Combining the two results gives:
  \[
    \frac{d}{d\lambda} \frac{\partial L}{\partial \dot{x}^\gamma} = \frac{\partial L}{\partial x^\gamma} \Rightarrow L \eta_{\gamma\nu} \frac{d^2x^\nu}{d\tau^2} = 0
  \]
  Because $\eta_{\gamma\nu}$ is non-zero and $L$ is non-zero for timelike curves, the condition for $\delta \tau$ to be zero is:
  \[
    \frac{d^2x^\nu}{d\tau^2} = 0
  \]
  We know that this is a maximum not a minimum because the worldline is timelike,
  meaning:
  \[
    d\tau^2 < 0
  \]
\end{enumerate}
\chapter{Incompatibility of Gravity and Special Relativity}
\begin{enumerate}
  \item \textbf{SCALAR GRAVITATIONAL THEORY, \Phi}

    \textbf{A. Consider the variationa principle $\delta I = 0$, where}
    \[
      I = -m \int e^\Phi \Big( - \eta_{\alpha\beta} \frac{dz^\alpha}{d\lambda} \frac{dz^\beta}{d\lambda} \Big)^{1 / 2} d\lambda
    \]
    \textbf{Here $m = \text{(rest mass)}$ and $z^\alpha(\lambda)=\text{parameterized world line}$
    for a test particle in the scalar gravitational field $\Phi$. By varying the
    particle's world line, derive differential equtions governing the particle's
    motion. Write them using the particle's proper time as the path parameter,}
    \[
      d\tau = \Big( - \eta_{\alpha\beta} \frac{dz^\alpha}{d\lambda} \frac{dz^\beta}{d\lambda} \Big)^{1 / 2} d\lambda
    \]
    \textbf{so that $u^\alpha = dz^\alpha / d\tau$ satisfies $u^\alpha u^\beta \eta_{\alpha\beta} = -1$}
\end{enumerate}
\chapter{Differential Geometry: An Overview}
\chapter{Differential Topology}
\begin{enumerate}
  \item \textbf{COMPONENT MANIPULATIONS}

  \textbf{Derive equations:}
  \begin{equation}
    u^\alpha = \langle \boldsymbol{\omega}^\alpha \boldsymbol{u} \rangle
  \end{equation}
  \begin{equation}
    \sigma_\beta = \langle \boldsymbol{\sigma}, \boldsymbol{e}_\beta \rangle
  \end{equation}
  \begin{equation}
    \langle \boldsymbol{\sigma}, \boldsymbol{u} \rangle = \sigma_\alpha u^\alpha
  \end{equation}
  \begin{equation}
    \boldsymbol{\omega}^{\alpha'} = L^{\alpha'}_\beta \boldsymbol{\omega}^\beta
  \end{equation}
  \begin{equation}
    \sigma_{\alpha'} = \sigma_\beta L^\beta_{\alpha'}
  \end{equation}
  Consider:
  \[
    \begin{align}
      \langle \boldsymbol{\omega}^\alpha, \boldsymbol{u} \rangle &= u^\beta \langle \boldsymbol{\omega}^\alpha, \boldsymbol{e}_\beta \rangle \\
                                                                 &= u^\beta \delta^\alpha_\beta \\
                                                                 &= u^\alpha
    \end{align}
  \]
  \[
    \begin{align}
      \langle \boldsymbol{\sigma}, \boldsymbol{e}_\beta \rangle &= \sigma_\alpha \langle \boldsymbol{\omega}^\alpha, \boldsymbol{e_\beta} \rangle \\
      &= \sigma_\alpha \delta^\alpha_\beta \\
      &= \sigma_\beta
    \end{align}
  \]
  \[
  \]
  \[
    \begin{align}
      \langle \boldsymbol{\sigma}, \boldsymbol{u} \rangle &= \sigma_\alpha u^\beta \langle \boldsymbol{\omega}^\alpha, \boldsymbol{e}_\alpha \rangle\\
                                                          &= \sigma_\alpha u^\beta \delta^\alpha_\beta \\
                                                          &= \sigma_\beta u^\beta
    \end{align}
  \]
  \[
    \begin{align}
    \boldsymbol{\sigma} &= \sigma_{\alpha'}\boldsymbol{\omega}^{\alpha'}\\
                        &= \sigma_\beta \boldsymbol{\omega}^\beta \\
                        &= \sigma_\gamma \delta^\gamma_\beta \boldsymbol{\omega}^\beta \\
                        &= \sigma_\gamma L^\gamma_{\alpha'} L^{\alpha'}_\beta \boldsymbol{\omega}^\beta \\
    \end{align}
  \]
  From inspection of components, it is clear that:
  \[
    \sigma_{\alpha'} = \sigma_\gamma L^\gamma_{\alpha'}
  \]
  \[
    \boldsymbol{\omega}^{\alpha'} = L^{\alpha'}_\beta \boldsymbol{\omega}^\beta
  \]
\item \textbf{COMPONENTS OF GRADIENT, AND DUALITY OF CORDINATE BASES}

  \textbf{In an arbitrary basis, define $f_{,\alpha}$ by the expansion:}
  \[
    \boldsymbol{d}f = f_{,\alpha} \boldsymbol{\omega}^\alpha
  \]
  \textbf{Then combine equations:}
  \[
    \sigma_\beta = \langle \boldsymbol{\sigma}, \boldsymbol{e}_\beta \rangle
  \]
  \textbf{and:}
  \[
    \langle \boldsymbol{d} f, \boldsymbol{u} \rangle = \partial_{\boldsymbol{u}} f = \boldsymbol{u}[f]
  \]
  \textbf{to obtain the method:}
  \[
    f_{,\alpha} = \partial_{\alpha} f = \boldsymbol{e}_\alpha [f]
  \]
  \textbf{Finally, combine equations:}
  \[
    \langle \boldsymbol{d} f, \boldsymbol{u} \rangle = \partial_{\boldsymbol{u}} f = \boldsymbol{u}[f]
  \]
  \textbf{and:}
  \[
    f_{,\alpha} = \partial_{\alpha} f = \boldsymbol{e}_\alpha [f]
  \]
  \textbf{to show that the bases $\boldsymbol{d}x^\alpha$ and $\partial / \partial x^\beta$ are the duals of each other.}
  Consider:
  \[
    \begin{align}
      \langle \boldsymbol{d}f, \boldsymbol{u} \rangle &= f_{,\alpha} u^\beta \langle \boldsymbol{\omega}^\alpha, \boldsymbol{e}_\beta \rangle \\
                                                      &= f_{,\alpha} u^\beta \delta^\alpha_\beta \\
                                                      &= f_{,\alpha} u^\alpha
    \end{align}
  \]
\end{enumerate}
\end{document}
